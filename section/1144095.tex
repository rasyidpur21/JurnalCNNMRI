\section{Jurnal CNN MRI}
\subsection{2015}
\begin{enumerate}
  \item Judul : \textit{Deep Convolutional Neural Networks for the Segmentation of Gliomas in Multi-sequence MRI}
  \item Author : \textit{Pereira, S{\'e}rgio and Pinto, Adriano and Alves, Victor and Silva, Carlos A}
  \item Journal : \textit{booktitle = BrainLes 2015, pages = 131--143, year = 2015, organization = Springer}
  \item Masalah : \textit{to segment brain tumors using a Deep Convolutional Neural Network. Neural Networks are known to suffer from overfitting.}
  \item Kontribusi : \textit{In this paper, we presented a CNN to segment brain tumors in MRI. Excluding when the user needs to identify the tumor grade, all steps in the processing pipeline are automatic. Although simple, this architecture shows promising
Deep CNNs for the Segmentation of Gliomas in Multi-sequence MRI 141
results, with space for further developments, especially in the Core region and
segmentation of LGG; in the Challenge data set the proposed method was ranked
in the second position. As future work, we want to make a totally grade independent method, possibly through a joint LGG/HGG training or an automatic
grade identification procedure before segmentation.}

  \item Metode / Solusi : \textit{use Dropout, Leaky Rectifier Linear Units and small convolutional kernels.} 
  \item Hasil : \textit{in each grade of the Training data set were used for training the CNN, with
the exception of six validation patients in each grade. To train the CNNs we
extracted around 4,000,000 training patches of HGG and 1,800,000 of LGG, and
we used mini-batches of 128 training samples. However, the number of training
patches was 4 times bigger due to the data augmentation. Observing Fig. 4, the
segmentations seem coherent with the expected tumor tissues, for example, the
enhanced tumor portions appear delineated following the enhancing parts in
T1c. Also, the complete tumor appears to be well delineated, when comparing
with the FLAIR and T2 sequences, where the edema is hyperintense.}
  \item Batasan : \textit{used mini-batches of 128 training samples}
\end{enumerate}